\documentclass[10pt]{article}% ===> this file was generated automatically by noweave --- better not edit it
\usepackage{noweb,setspace,amsmath,amsthm,amssymb,microtype,ragged2e,ae}
\usepackage[round]{natbib}
\usepackage[top=1in,left=1.75in,right=1.75in,bottom=1.25in]{geometry}
\frenchspacing
\DeclareMicrotypeAlias{cmor}{cmr}
\DisableLigatures{family=tt*}
\noweboptions{longxref}
\let\nowebsize=\small
% \def\nwendcode{\endtrivlist \endgroup}
% \let\nwdocspar=\par
\bibliographystyle{abbrvnat}
\newcommand\citepos[2][]{\citeauthor{#2}'s \citeyearpar[#1]{#2}}

\newcommand{\dmw}{\textsc{dmw}}
\newcommand{\gnu}{\textsc{gnu}}
\newcommand{\mds}{\textsc{mds}}
\newcommand{\oos}{\textsc{oos}}

\renewcommand{\Re}{\ensuremath{\mathbb{R}}}
\DeclareMathOperator{\tr}{tr}
\DeclareMathOperator{\WN}{WN}
\DeclareMathOperator{\E}{E}
\DeclareMathOperator{\var}{var}
\DeclareMathOperator{\N}{N}
\DeclareMathOperator*{\argmin}{arg\,min}
\DeclareMathOperator*{\plim}{plim}
\DeclareMathOperator{\eig}{\lambda_{\max}}
\DeclareMathOperator{\eigl}{\lambda_{\min}}
\DeclareMathOperator{\p}{Pr}
\DeclareMathOperator{\ind}{1}

\makeatletter
\renewcommand\section{\@startsection%
{section}{1}{0pt}{-\baselineskip}{0.5\baselineskip}%
{\normalfont\normalsize\bfseries\large\raggedright}}
\renewcommand\subsection{\@startsection%
{subsection}{2}{0pt}{-0.5\baselineskip}{0.5\baselineskip}%
{\normalfont\normalsize\bfseries\small\raggedright}}
\makeatother

\begin{document}
\input{version}
\title{Implementation details of the `oosanalysis' package.}

\author{Gray Calhoun\footnote{email:
    \texttt{gcalhoun@iastate.edu}}\\Iowa State University}

\maketitle

\tableofcontents

\section{Out-of-sample comparisons}

This section describes the code for the actual test statistics.

\subsection{Diebold-Mariano-West test statistic}
Probably the most important of the \oos\ statistics is the
Diebold-Mariano-West test \citep{DiM:95,Wes:96}.  For now, I've just
written up a bare-bones version of the statistic that requires the
user to explictly calculate \citepos{Wes:96} variables and pass them
to the function.  The {\Tt{}\nwlinkedidentq{dmw{\_}mse}{NW29Cep0-3NQVMk-3}\nwendquote} function provides a more convenient
wrapper for the most common use-case.  The function {\Tt{}\nwlinkedidentq{mixedwindow}{NW29Cep0-3NQVMk-4}\nwendquote}
also does provide a higher-level function that then calls this
function, so it might be a good template if one wants to code up
others.

The function {\Tt{}\nwlinkedidentq{dmw{\_}calculation}{NW29Cep0-3NQVMk-1}\nwendquote} estimates the mean and variance of
the \oos\ process $f_{t}(\beta)$.  In \citet{WeM:98}, the recursive
window coefficient estimator can be written in the form
\begin{equation}
  \label{eq:1}
  \hat{\beta}_{t} = B_{t} \tfrac{1}{T} \sum_{s=1}^{t}
  h_{t}.\footnote{The notation for the recursive window is slightly
    different than \citet{Wes:96} uses.}
\end{equation}
Changing the formula for the fixed and rolling window estimators is
straightforward.  The asymptotic variance of the \oos\ average is
\begin{equation}
  \label{eq:2}
  S_{ff} + (S_{fh} B' F' + F B S_{fh}') \lambda_{fh} + \lambda_{hh}
  F B S_{hh} B' F'
\end{equation}
where $B = \plim B_{t}$, $F = \E \tfrac{\partial}{\partial \beta}
f_{t}(\beta)$, $S$ is the long-run variance of $(f_{t}, h_{t})$, and
$\lambda$ is a vector of scale factors that depend on the window used.

\nwfilename{oosanalysis/noweb/implementation.rnw}\nwbegincode{1}\sublabel{NW29Cep0-3NQVMk-1}\nwmargintag{{\nwtagstyle{}\subpageref{NW29Cep0-3NQVMk-1}}}\moddef{dmw.R~{\nwtagstyle{}\subpageref{NW29Cep0-3NQVMk-1}}}\endmoddef\nwstartdeflinemarkup\nwprevnextdefs{\relax}{NW29Cep0-3NQVMk-2}\nwenddeflinemarkup
\LA{}Copyright and licensing~{\nwtagstyle{}\subpageref{NW29Cep0-4UiXc2-1}}\RA{}
\nwlinkedidentc{dmw_calculation}{NW29Cep0-3NQVMk-1} <- function(f, h, R, vcv, tBtF = NULL, pi = noos / R,
                            window = c("recursive", "rolling", "fixed")) \{
  noos <- length(f)
  lambda <- \nwlinkedidentc{dmw_lambda}{NW29Cep0-3NQVMk-2}(pi, window)
  htBtF <- if (is.null(tBtF)) \{
    rep(0, noos) 
  \} else \{
    tcrossprod(as.matrix(h), matrix(tBtF, nrow = 1))
  \}
  S <- vcv(cbind(f, htBtF))
  return(list(mu = mean(f), avar = S[1,1] + lambda$fh * (S[1,2] + S[2,1]) 
                                   + lambda$hh * S[2,2]))
\}
\nwindexdefn{\nwixident{dmw{\_}calculation}}{dmw:uncalculation}{NW29Cep0-3NQVMk-1}\eatline
\nwalsodefined{\\{NW29Cep0-3NQVMk-2}\\{NW29Cep0-3NQVMk-3}\\{NW29Cep0-3NQVMk-4}\\{NW29Cep0-3NQVMk-5}\\{NW29Cep0-3NQVMk-6}}\nwnotused{dmw.R}\nwidentdefs{\\{{\nwixident{dmw{\_}calculation}}{dmw:uncalculation}}}\nwidentuses{\\{{\nwixident{dmw{\_}lambda}}{dmw:unlambda}}}\nwindexuse{\nwixident{dmw{\_}lambda}}{dmw:unlambda}{NW29Cep0-3NQVMk-1}\nwendcode{}\nwbegindocs{2}\nwdocspar
The function {\Tt{}\nwlinkedidentq{dmw{\_}lambda}{NW29Cep0-3NQVMk-2}\nwendquote} calculates the scale terms used to
calculate the asymptotic covariance matrix of the \oos\ average, $\bar
f$.  \citet{Wes:96} derives the formula for the recursive window and
\citet{WeM:98} for the rolling and fixed windows.
\nwenddocs{}\nwbegincode{3}\sublabel{NW29Cep0-3NQVMk-2}\nwmargintag{{\nwtagstyle{}\subpageref{NW29Cep0-3NQVMk-2}}}\moddef{dmw.R~{\nwtagstyle{}\subpageref{NW29Cep0-3NQVMk-1}}}\plusendmoddef\nwstartdeflinemarkup\nwprevnextdefs{NW29Cep0-3NQVMk-1}{NW29Cep0-3NQVMk-3}\nwenddeflinemarkup
\nwlinkedidentc{dmw_lambda}{NW29Cep0-3NQVMk-2} <- function(pi, window = c("recursive", "rolling", "fixed")) \{
  window <- match.arg(window)
  stopifnot(pi >= 0)
  if (window == "recursive") \{
    if (pi >= Inf) \{
      lambda.fh <- 1
    \} else if (pi > 0) \{
      lambda.fh <- 1 - log(1 + pi)/pi
    \} else \{
      lambda.fh <- 0
    \}
    lambda.hh <- 2 * lambda.fh
  \} else if (window == "fixed") \{
    lambda.fh <- 0
    lambda.hh <- pi
  \} else if (window == "rolling" && pi <= 1) \{
    lambda.fh <- pi/2
    lambda.hh <- pi - pi^2 / 3
  \} else if (window == "rolling" && pi > 1) \{
    lambda.fh <- 1 - 1/(2*pi)
    lambda.hh <- 1 - 1/(3*pi)
  \}
  return(list(fh = lambda.fh, hh = lambda.hh))
\}
\nwindexdefn{\nwixident{dmw{\_}lambda}}{dmw:unlambda}{NW29Cep0-3NQVMk-2}\eatline
\nwidentdefs{\\{{\nwixident{dmw{\_}lambda}}{dmw:unlambda}}}\nwendcode{}\nwbegindocs{4}\nwdocspar
\nwenddocs{}\nwbegincode{5}\sublabel{NW29Cep0-3NQVMk-3}\nwmargintag{{\nwtagstyle{}\subpageref{NW29Cep0-3NQVMk-3}}}\moddef{dmw.R~{\nwtagstyle{}\subpageref{NW29Cep0-3NQVMk-1}}}\plusendmoddef\nwstartdeflinemarkup\nwprevnextdefs{NW29Cep0-3NQVMk-2}{NW29Cep0-3NQVMk-4}\nwenddeflinemarkup
\nwlinkedidentc{dmw_mse}{NW29Cep0-3NQVMk-3} <- function(null, alt, dataset, R, vcv = var,
                    window = c("recursive", "rolling", "fixed")) \{

  y <- \nwlinkedidentc{extract_target}{NW29Cep0-F2wuL-1}(null, dataset)
  X <- \nwlinkedidentc{extract_predictors}{NW29Cep0-F2wuL-1}(null, dataset)
  forecasts.null <- \nwlinkedidentc{recursive_forecasts}{NW29Cep0-15rBrk-1}(null, dataset, R, window)
  forecasts.alt <- \nwlinkedidentc{recursive_forecasts}{NW29Cep0-15rBrk-1}(alt, dataset, R, window)
  
  nobs <- length(y)
  noos <- length(forecasts.null)
  oos <- seq.int(to = nobs, length = noos)

  errors.null <- y[oos] - forecasts.null
  errors.alt  <- y[oos] - forecasts.alt

  estimates <- \nwlinkedidentc{dmw_calculation}{NW29Cep0-3NQVMk-1}(f = errors.null^2 - errors.alt^2,
                               h = errors.null * X[oos,,drop = FALSE],
                               R = nobs - noos, vcv = vcv, 
                               pi = noos / (nobs - noos), window = "recursive")
  estimates$tstat <- with(estimates, mu * sqrt((nobs - R) / avar))
  estimates$pvalue <- pnorm(estimates$tstat, lower.tail = FALSE)
  return(estimates)
\}
\nwindexdefn{\nwixident{dmw{\_}mse}}{dmw:unmse}{NW29Cep0-3NQVMk-3}\eatline
\nwidentdefs{\\{{\nwixident{dmw{\_}mse}}{dmw:unmse}}}\nwidentuses{\\{{\nwixident{dmw{\_}calculation}}{dmw:uncalculation}}\\{{\nwixident{extract{\_}predictors}}{extract:unpredictors}}\\{{\nwixident{extract{\_}target}}{extract:untarget}}\\{{\nwixident{recursive{\_}forecasts}}{recursive:unforecasts}}}\nwindexuse{\nwixident{dmw{\_}calculation}}{dmw:uncalculation}{NW29Cep0-3NQVMk-3}\nwindexuse{\nwixident{extract{\_}predictors}}{extract:unpredictors}{NW29Cep0-3NQVMk-3}\nwindexuse{\nwixident{extract{\_}target}}{extract:untarget}{NW29Cep0-3NQVMk-3}\nwindexuse{\nwixident{recursive{\_}forecasts}}{recursive:unforecasts}{NW29Cep0-3NQVMk-3}\nwendcode{}\nwbegindocs{6}\nwdocspar
\subsection{\protect\citet{Mcc:07} critical values}
\citet{Mcc:07} derives critical values for the oos-$t$ test under the
assumption that the benchmark model is nested in the alternative
candidate model.  This setup introduces a degeneracy under the null so
the test statistic is no longer asymptotically normal.  \citet{Mcc:07}
presents a table of new critical values for the statistic, and the
{\Tt{}\nwlinkedidentq{mccracken{\_}criticalvalue}{NW29Cep0-HXXXl-1}\nwendquote} function is a simple wrapper that returns
the appropriate critical value.  The underlying data are defined in
the Appendix.

\nwenddocs{}\nwbegincode{7}\sublabel{NW29Cep0-HXXXl-1}\nwmargintag{{\nwtagstyle{}\subpageref{NW29Cep0-HXXXl-1}}}\moddef{mccracken.R~{\nwtagstyle{}\subpageref{NW29Cep0-HXXXl-1}}}\endmoddef\nwstartdeflinemarkup\nwenddeflinemarkup
\LA{}Define faster \code{}findInterval\edoc{} function~{\nwtagstyle{}\subpageref{NW29Cep0-2cq75d-1}}\RA{}
\nwlinkedidentc{mccracken_criticalvalue}{NW29Cep0-HXXXl-1} <- 
  function(pi, k2, confidence, window = c("recursive", "rolling", "fixed")) \{
    window <- match.arg(window)
    confidenceIndex <- match(confidence, c(0.99, 0.95, 0.90))
    if (pi <= \nwlinkedidentc{PiBounds}{NW29Cep0-18ASnA-1}[1]) \{
      return(\nwlinkedidentc{McCrackenData}{NW29Cep0-18ASnA-1}[[window]][1, confidenceIndex, k2])
    \} else if (pi >= \nwlinkedidentc{PiBounds}{NW29Cep0-18ASnA-1}[2]) \{
      return(\nwlinkedidentc{McCrackenData}{NW29Cep0-18ASnA-1}[[window]][length(\nwlinkedidentc{PiIntervals}{NW29Cep0-18ASnA-1}), confidenceIndex, k2])
    \} else \{
      leftIndex <- \nwlinkedidentc{findInterval_presorted}{NW29Cep0-2cq75d-1}(pi, \nwlinkedidentc{PiIntervals}{NW29Cep0-18ASnA-1})
      piLeft <- \nwlinkedidentc{PiIntervals}{NW29Cep0-18ASnA-1}[leftIndex]
      wRight <- (pi - piLeft) / (\nwlinkedidentc{PiIntervals}{NW29Cep0-18ASnA-1}[leftIndex + 1] - piLeft)
      critLeft <- \nwlinkedidentc{McCrackenData}{NW29Cep0-18ASnA-1}[[window]][leftIndex, confidenceIndex, k2]
      critRight <- \nwlinkedidentc{McCrackenData}{NW29Cep0-18ASnA-1}[[window]][leftIndex + 1, confidenceIndex, k2]
      return((1 - wRight) * critLeft + wRight * critRight)
    \}
  \}
\nwindexdefn{\nwixident{mccracken{\_}criticalvalue}}{mccracken:uncriticalvalue}{NW29Cep0-HXXXl-1}\eatline
\nwnotused{mccracken.R}\nwidentdefs{\\{{\nwixident{mccracken{\_}criticalvalue}}{mccracken:uncriticalvalue}}}\nwidentuses{\\{{\nwixident{findInterval{\_}presorted}}{findInterval:unpresorted}}\\{{\nwixident{McCrackenData}}{McCrackenData}}\\{{\nwixident{PiBounds}}{PiBounds}}\\{{\nwixident{PiIntervals}}{PiIntervals}}}\nwindexuse{\nwixident{findInterval{\_}presorted}}{findInterval:unpresorted}{NW29Cep0-HXXXl-1}\nwindexuse{\nwixident{McCrackenData}}{McCrackenData}{NW29Cep0-HXXXl-1}\nwindexuse{\nwixident{PiBounds}}{PiBounds}{NW29Cep0-HXXXl-1}\nwindexuse{\nwixident{PiIntervals}}{PiIntervals}{NW29Cep0-HXXXl-1}\nwendcode{}\nwbegindocs{8}\nwdocspar
The {\Tt{}findInterval\nwendquote} function in the base package implements a binary
search, but performs a few consistency checks that can slow down the
search.  I've added a version of the function without those
consistency checks that is faster, but may cause problems if it's used
recklessly.

\nwenddocs{}\nwbegincode{9}\sublabel{NW29Cep0-2cq75d-1}\nwmargintag{{\nwtagstyle{}\subpageref{NW29Cep0-2cq75d-1}}}\moddef{Define faster \code{}findInterval\edoc{} function~{\nwtagstyle{}\subpageref{NW29Cep0-2cq75d-1}}}\endmoddef\nwstartdeflinemarkup\nwusesondefline{\\{NW29Cep0-HXXXl-1}}\nwenddeflinemarkup
\nwlinkedidentc{findInterval_presorted}{NW29Cep0-2cq75d-1} <- 
 function(x, vec, rightmost.closed = FALSE, all.inside = FALSE) \{
    nx <- length(x)
    index <- integer(nx)
    .C("find_interv_vec", xt = as.double(vec), n = length(vec), 
       x = as.double(x), nx = nx, as.logical(rightmost.closed), 
       as.logical(all.inside), index, DUP = FALSE, NAOK = TRUE, 
       PACKAGE = "base")
    return(index)
\}
\nwindexdefn{\nwixident{findInterval{\_}presorted}}{findInterval:unpresorted}{NW29Cep0-2cq75d-1}\eatline
\nwused{\\{NW29Cep0-HXXXl-1}}\nwidentdefs{\\{{\nwixident{findInterval{\_}presorted}}{findInterval:unpresorted}}}\nwendcode{}\nwbegindocs{10}\nwdocspar
\subsection{\citet{ClW:06,ClW:07} test statistic}
The next statisic is developed in \citet{ClW:06,ClW:07}, who propose
using a fixed-length rolling window to get an asymptotically normal
statistic \citep[as in][]{Giw:06}, but add a correction to the \oos\
average so that it's centered correctly for testing the null that the
benchmark is an \mds.  I implement it as two functions.  One to
estimate the \oos\ forecasts for given data and models.  And one to
construct the statistic.

\nwenddocs{}\nwbegincode{11}\sublabel{NW29Cep0-2ARW6h-1}\nwmargintag{{\nwtagstyle{}\subpageref{NW29Cep0-2ARW6h-1}}}\moddef{clarkwest.R~{\nwtagstyle{}\subpageref{NW29Cep0-2ARW6h-1}}}\endmoddef\nwstartdeflinemarkup\nwprevnextdefs{\relax}{NW29Cep0-2ARW6h-2}\nwenddeflinemarkup
\LA{}Copyright and licensing~{\nwtagstyle{}\subpageref{NW29Cep0-4UiXc2-1}}\RA{}
\nwlinkedidentc{clarkwest}{NW29Cep0-2ARW6h-1} <- function(null, alt, dataset, R, vcv = var,
                      window = c("rolling", "recursive", "fixed"))
  \nwlinkedidentc{clarkwest_calculation}{NW29Cep0-2ARW6h-2}(
    target        = \nwlinkedidentc{extract_target}{NW29Cep0-F2wuL-1}(null, dataset[-seq_len(R),,drop = FALSE]),
    null.forecast = \nwlinkedidentc{recursive_forecasts}{NW29Cep0-15rBrk-1}(null, dataset, R, window),
    alt.forecast  = \nwlinkedidentc{recursive_forecasts}{NW29Cep0-15rBrk-1}(alt, dataset, R, window),
    vcv)
\nwindexdefn{\nwixident{clarkwest}}{clarkwest}{NW29Cep0-2ARW6h-1}\eatline
\nwalsodefined{\\{NW29Cep0-2ARW6h-2}}\nwnotused{clarkwest.R}\nwidentdefs{\\{{\nwixident{clarkwest}}{clarkwest}}}\nwidentuses{\\{{\nwixident{clarkwest{\_}calculation}}{clarkwest:uncalculation}}\\{{\nwixident{extract{\_}target}}{extract:untarget}}\\{{\nwixident{recursive{\_}forecasts}}{recursive:unforecasts}}}\nwindexuse{\nwixident{clarkwest{\_}calculation}}{clarkwest:uncalculation}{NW29Cep0-2ARW6h-1}\nwindexuse{\nwixident{extract{\_}target}}{extract:untarget}{NW29Cep0-2ARW6h-1}\nwindexuse{\nwixident{recursive{\_}forecasts}}{recursive:unforecasts}{NW29Cep0-2ARW6h-1}\nwendcode{}\nwbegindocs{12}\nwdocspar
\nwenddocs{}\nwbegincode{13}\sublabel{NW29Cep0-2ARW6h-2}\nwmargintag{{\nwtagstyle{}\subpageref{NW29Cep0-2ARW6h-2}}}\moddef{clarkwest.R~{\nwtagstyle{}\subpageref{NW29Cep0-2ARW6h-1}}}\plusendmoddef\nwstartdeflinemarkup\nwprevnextdefs{NW29Cep0-2ARW6h-1}{\relax}\nwenddeflinemarkup
\nwlinkedidentc{clarkwest_calculation}{NW29Cep0-2ARW6h-2} <- function(target, null.forecast, 
                                  alt.forecast, vcv) \{
  P <- length(target)
  oos.sequence <- \{(target - null.forecast)^2 - 
                   (target - alt.forecast)^2 + 
                   (null.forecast - alt.forecast)^2\}
  mu <- mean(oos.sequence)
  avar <- vcv(oos.sequence)
  return(list(mu = mu, avar = avar, 
              pvalue = pnorm(sqrt(P) * mu, 0, sqrt(avar), FALSE)))
\}
\nwindexdefn{\nwixident{clarkwest{\_}calculation}}{clarkwest:uncalculation}{NW29Cep0-2ARW6h-2}\eatline
\nwidentdefs{\\{{\nwixident{clarkwest{\_}calculation}}{clarkwest:uncalculation}}}\nwendcode{}\nwbegindocs{14}\nwdocspar
\subsection{Mixed window test statistics}

This subsection implements the statistic proposed by \citet{Cal:11b}.
This statistic uses a recursive window for the benchmark and a
fixed-length rolling or fixed window for the alternative.  The
function {\Tt{}\nwlinkedidentq{mixedwindow}{NW29Cep0-3NQVMk-4}\nwendquote} is basically a wrapper for
{\Tt{}\nwlinkedidentq{mixedwindow{\_}calculation}{NW29Cep0-3NQVMk-5}\nwendquote}.
\nwenddocs{}\nwbegincode{15}\sublabel{NW29Cep0-3NQVMk-4}\nwmargintag{{\nwtagstyle{}\subpageref{NW29Cep0-3NQVMk-4}}}\moddef{dmw.R~{\nwtagstyle{}\subpageref{NW29Cep0-3NQVMk-1}}}\plusendmoddef\nwstartdeflinemarkup\nwprevnextdefs{NW29Cep0-3NQVMk-3}{NW29Cep0-3NQVMk-5}\nwenddeflinemarkup
\LA{}Copyright and licensing~{\nwtagstyle{}\subpageref{NW29Cep0-4UiXc2-1}}\RA{}
\nwlinkedidentc{mixedwindow}{NW29Cep0-3NQVMk-4} <- function(null, alt, dataset, R, vcv = var,
                      window = c("rolling", "fixed"), pimethod = "estimate") \{
  nobs <- nrow(dataset)
  estimates <-
    \nwlinkedidentc{mixedwindow_calculation}{NW29Cep0-3NQVMk-5}(y = \nwlinkedidentc{extract_target}{NW29Cep0-F2wuL-1}(null, dataset),
                            X = \nwlinkedidentc{extract_predictors}{NW29Cep0-F2wuL-1}(null, dataset),
                            \nwlinkedidentc{recursive_forecasts}{NW29Cep0-15rBrk-1}(null, dataset, R, "recursive"),
                            \nwlinkedidentc{recursive_forecasts}{NW29Cep0-15rBrk-1}(alt, dataset, R, window),
                            vcv = vcv, pimethod = pimethod)
  estimates$tstat <- with(estimates, mu * sqrt((nobs - R) / avar))
  estimates$pvalue <- pnorm(estimates$tstat, lower.tail = FALSE)
  return(estimates)
\}
\nwindexdefn{\nwixident{mixedwindow}}{mixedwindow}{NW29Cep0-3NQVMk-4}\eatline
\nwidentdefs{\\{{\nwixident{mixedwindow}}{mixedwindow}}}\nwidentuses{\\{{\nwixident{extract{\_}predictors}}{extract:unpredictors}}\\{{\nwixident{extract{\_}target}}{extract:untarget}}\\{{\nwixident{mixedwindow{\_}calculation}}{mixedwindow:uncalculation}}\\{{\nwixident{recursive{\_}forecasts}}{recursive:unforecasts}}}\nwindexuse{\nwixident{extract{\_}predictors}}{extract:unpredictors}{NW29Cep0-3NQVMk-4}\nwindexuse{\nwixident{extract{\_}target}}{extract:untarget}{NW29Cep0-3NQVMk-4}\nwindexuse{\nwixident{mixedwindow{\_}calculation}}{mixedwindow:uncalculation}{NW29Cep0-3NQVMk-4}\nwindexuse{\nwixident{recursive{\_}forecasts}}{recursive:unforecasts}{NW29Cep0-3NQVMk-4}\nwendcode{}\nwbegindocs{16}{\Tt{}\nwlinkedidentq{mixedwindow{\_}calculation}{NW29Cep0-3NQVMk-5}\nwendquote} constructs the \oos\ processes and then 
them to the {\Tt{}\nwlinkedidentq{dmw{\_}calculation}{NW29Cep0-3NQVMk-1}\nwendquote} function to generate the
final statistics.  It might seem a little silly to have
{\Tt{}\nwlinkedidentq{mixedwindow}{NW29Cep0-3NQVMk-4}\nwendquote} and {\Tt{}\nwlinkedidentq{mixedwindow{\_}calculation}{NW29Cep0-3NQVMk-5}\nwendquote} as distinct functions,
but we reuse {\Tt{}\nwlinkedidentq{mixedwindow{\_}calculation}{NW29Cep0-3NQVMk-5}\nwendquote} in the bootstrap.

\nwenddocs{}\nwbegincode{17}\sublabel{NW29Cep0-3NQVMk-5}\nwmargintag{{\nwtagstyle{}\subpageref{NW29Cep0-3NQVMk-5}}}\moddef{dmw.R~{\nwtagstyle{}\subpageref{NW29Cep0-3NQVMk-1}}}\plusendmoddef\nwstartdeflinemarkup\nwprevnextdefs{NW29Cep0-3NQVMk-4}{NW29Cep0-3NQVMk-6}\nwenddeflinemarkup
\nwlinkedidentc{mixedwindow_calculation}{NW29Cep0-3NQVMk-5} <- function(y, X, forecasts.null, forecasts.alt, vcv,
                                    pimethod = "estimate") \{
  X <- as.matrix(X)
  nobs <- length(y)
  noos <- length(forecasts.null)
  oos <- seq.int(to = nobs, length = noos)
  errors.null <- y[oos] - forecasts.null
  errors.alt  <- y[oos] - forecasts.alt
  forecast.differences <- forecasts.null - forecasts.alt

  pivalue <- switch(pimethod, estimate = noos / (nobs - noos), theory = Inf,
                    stop("That choice of 'pi' is unsupported."))
  \nwlinkedidentc{dmw_calculation}{NW29Cep0-3NQVMk-1}(f = errors.null^2 - errors.alt^2 + forecast.differences^2,
                  h = errors.null * X[oos,,drop = FALSE],
                  R = nobs - noos, vcv = vcv,
                  tBtF = 2 * solve(crossprod(X) / nobs,
                    colMeans(forecast.differences * X[oos,,drop=FALSE])),
                  pi = pivalue, window = "recursive")
\}
\nwindexdefn{\nwixident{mixedwindow{\_}calculation}}{mixedwindow:uncalculation}{NW29Cep0-3NQVMk-5}\eatline
\nwidentdefs{\\{{\nwixident{mixedwindow{\_}calculation}}{mixedwindow:uncalculation}}}\nwidentuses{\\{{\nwixident{dmw{\_}calculation}}{dmw:uncalculation}}}\nwindexuse{\nwixident{dmw{\_}calculation}}{dmw:uncalculation}{NW29Cep0-3NQVMk-5}\nwendcode{}\nwbegindocs{18}\nwdocspar
The {\Tt{}\nwlinkedidentq{mixedbootstrap}{NW29Cep0-3NQVMk-6}\nwendquote} function uses {\Tt{}\nwlinkedidentq{mixedwindow{\_}calculation}{NW29Cep0-3NQVMk-5}\nwendquote} to
implement \citepos{Cal:11b} bootstrap.  I implement the bootstrap by
generating random indices rather than actually randomizing the data,
since we have several distinct sequences that we need to bootstrap
jointly (I assume that this is the usual way people do it).  The local
function {\Tt{}\nwlinkedidentq{rbootindex}{NW29Cep0-3SDs62-1}\nwendquote} takes no arguments, but is defined to generate
a length {\Tt{}nobs\nwendquote} vector with blocks of length {\Tt{}blocklength\nwendquote} that
induces the moving blocks \citep{Kun:89,LiS:92} or circular blocks
\citep{PoR:92} bootstraps.  
The {\Tt{}extendedindex\nwendquote} vector automatically
includes the first {\Tt{}R\nwendquote} observations so that the bootstrap population
mean is equal to the sample mean of the data (for the circular blocks
bootstrap).
\nwenddocs{}\nwbegincode{19}\sublabel{NW29Cep0-3NQVMk-6}\nwmargintag{{\nwtagstyle{}\subpageref{NW29Cep0-3NQVMk-6}}}\moddef{dmw.R~{\nwtagstyle{}\subpageref{NW29Cep0-3NQVMk-1}}}\plusendmoddef\nwstartdeflinemarkup\nwprevnextdefs{NW29Cep0-3NQVMk-5}{\relax}\nwenddeflinemarkup
\nwlinkedidentc{mixedbootstrap}{NW29Cep0-3NQVMk-6} <- function(null, alt.list, dataset, R, nboot, blocklength,
                           vcv = var, window = c("rolling", "fixed"),
                           bootstrap = c("moving", "circular", "stationary"),
                           pimethod = "estimate") \{
  \LA{}Define \code{}mixedbootstrap\edoc{} local variables~{\nwtagstyle{}\subpageref{NW29Cep0-3CIgww-1}}\RA{}
  \LA{}Calculate individual test statistics~{\nwtagstyle{}\subpageref{NW29Cep0-2j4ZdE-1}}\RA{}
  \LA{}Calculate individual bootstrap means~{\nwtagstyle{}\subpageref{nw@notdef}}\RA{}
  \LA{}Define block bootstrap function \code{}rbootindex\edoc{}~{\nwtagstyle{}\subpageref{NW29Cep0-3SDs62-1}}\RA{}
  bootstrap.replications <- replicate(nboot, \{
    bootindex <- \nwlinkedidentc{rbootindex}{NW29Cep0-3SDs62-1}()
    extendedindex <- c(seq_len(R), R + bootindex)
    sapply(forecasts.alt, function(alt) \{
      with(\nwlinkedidentc{mixedwindow_calculation}{NW29Cep0-3NQVMk-5}(y[extendedindex],
             X[extendedindex,,drop=FALSE],
             \nwlinkedidentc{recursive_forecasts}{NW29Cep0-15rBrk-1}(null, dataset[extendedindex,], R, "recursive"),
             alt[bootindex], vcv, pimethod),
           mu * sqrt(noos / avar))
  \})\})
  return(list(statistics = test.statistics,
              replications = bootstrap.replications - popmeans.boot))
\}
\nwindexdefn{\nwixident{mixedbootstrap}}{mixedbootstrap}{NW29Cep0-3NQVMk-6}\eatline
\nwidentdefs{\\{{\nwixident{mixedbootstrap}}{mixedbootstrap}}}\nwidentuses{\\{{\nwixident{mixedwindow{\_}calculation}}{mixedwindow:uncalculation}}\\{{\nwixident{rbootindex}}{rbootindex}}\\{{\nwixident{recursive{\_}forecasts}}{recursive:unforecasts}}}\nwindexuse{\nwixident{mixedwindow{\_}calculation}}{mixedwindow:uncalculation}{NW29Cep0-3NQVMk-6}\nwindexuse{\nwixident{rbootindex}}{rbootindex}{NW29Cep0-3NQVMk-6}\nwindexuse{\nwixident{recursive{\_}forecasts}}{recursive:unforecasts}{NW29Cep0-3NQVMk-6}\nwendcode{}\nwbegindocs{20}\nwdocspar
One nice feature of this bootstrap is that it's only necessary to
reestimate the benchmark model.  The alternative models are estimated
once, and their forecasts are resampled along with the original data.
The first step in the function is to define local variables and
generate these forecasts.
\nwenddocs{}\nwbegincode{21}\sublabel{NW29Cep0-3CIgww-1}\nwmargintag{{\nwtagstyle{}\subpageref{NW29Cep0-3CIgww-1}}}\moddef{Define \code{}mixedbootstrap\edoc{} local variables~{\nwtagstyle{}\subpageref{NW29Cep0-3CIgww-1}}}\endmoddef\nwstartdeflinemarkup\nwusesondefline{\\{NW29Cep0-3NQVMk-6}}\nwenddeflinemarkup
nobs <- nrow(dataset)
noos <- nobs - R
window <- match.arg(window)
bootstrap <- match.arg(bootstrap)
X <- \nwlinkedidentc{extract_predictors}{NW29Cep0-F2wuL-1}(null, dataset)
y <- \nwlinkedidentc{extract_target}{NW29Cep0-F2wuL-1}(null, dataset)
\nwused{\\{NW29Cep0-3NQVMk-6}}\nwidentuses{\\{{\nwixident{extract{\_}predictors}}{extract:unpredictors}}\\{{\nwixident{extract{\_}target}}{extract:untarget}}}\nwindexuse{\nwixident{extract{\_}predictors}}{extract:unpredictors}{NW29Cep0-3CIgww-1}\nwindexuse{\nwixident{extract{\_}target}}{extract:untarget}{NW29Cep0-3CIgww-1}\nwendcode{}\nwbegindocs{22}\nwdocspar

\nwenddocs{}\nwbegincode{23}\sublabel{NW29Cep0-2j4ZdE-1}\nwmargintag{{\nwtagstyle{}\subpageref{NW29Cep0-2j4ZdE-1}}}\moddef{Calculate individual test statistics~{\nwtagstyle{}\subpageref{NW29Cep0-2j4ZdE-1}}}\endmoddef\nwstartdeflinemarkup\nwusesondefline{\\{NW29Cep0-3NQVMk-6}}\nwenddeflinemarkup
forecasts.null <- \nwlinkedidentc{recursive_forecasts}{NW29Cep0-15rBrk-1}(null, dataset, R, "recursive")
forecasts.alt <- lapply(alt.list, function(m) \{
  \nwlinkedidentc{recursive_forecasts}{NW29Cep0-15rBrk-1}(m, dataset, R, window)
\})
makestats <- function(forecasts.null) sapply(forecasts.alt, function(alt) \{
  with(\nwlinkedidentc{mixedwindow_calculation}{NW29Cep0-3NQVMk-5}(y, X, forecasts.null, alt, vcv),
       mu * sqrt(noos / avar), pimethod)
\})
test.statistics <- makestats(forecasts.null)
popmeans.boot <- makestats(predict(null(dataset),
                           newdata = dataset[seq.int(from = R+1, length = noos),]))
\nwused{\\{NW29Cep0-3NQVMk-6}}\nwidentuses{\\{{\nwixident{mixedwindow{\_}calculation}}{mixedwindow:uncalculation}}\\{{\nwixident{recursive{\_}forecasts}}{recursive:unforecasts}}}\nwindexuse{\nwixident{mixedwindow{\_}calculation}}{mixedwindow:uncalculation}{NW29Cep0-2j4ZdE-1}\nwindexuse{\nwixident{recursive{\_}forecasts}}{recursive:unforecasts}{NW29Cep0-2j4ZdE-1}\nwendcode{}\nwbegindocs{24}\nwdocspar

The next two functions generate indices that induce the moving blocks
bootstrap or circular blocks bootstrap.  I prefer to have these as
separate fuctions rather than as a single function that takes the
bootstrap type as an argument.  I don't really see a reason to repeat
the logical operation to determine the bootstrap type every time we
want to generate a new bootstrap sample.  It seems like that
operation should be done only once (I realize that the time savings
are basically zero compared to the time it takes to do the estimation,
but still\dots).

These functions should almost certainly be moved to C code. 
\nwenddocs{}\nwbegincode{25}\sublabel{NW29Cep0-B4OnS-1}\nwmargintag{{\nwtagstyle{}\subpageref{NW29Cep0-B4OnS-1}}}\moddef{bootindex.R~{\nwtagstyle{}\subpageref{NW29Cep0-B4OnS-1}}}\endmoddef\nwstartdeflinemarkup\nwenddeflinemarkup
\LA{}Copyright and licensing~{\nwtagstyle{}\subpageref{NW29Cep0-4UiXc2-1}}\RA{}
\nwlinkedidentc{bootindex_movingblock}{NW29Cep0-B4OnS-1} <- function(nobs, blocklength) \{
  blockstarts <- sample(seq_len(nobs - blocklength + 1),
                        ceiling(nobs / blocklength),
                        replace = TRUE)
  indices <- 
    as.vector(sapply(blockstarts, function(s) s:(s + blocklength - 1)))[1:nobs]
  attr(indices, "starts") <- blockstarts
  attr(indices, "lengths") <- blocklength
  indices
\}

\nwlinkedidentc{bootindex_circularblock}{NW29Cep0-B4OnS-1} <- function(nobs, blocklength) \{
  blockstarts <- sample(seq_len(nobs), ceiling(nobs / blocklength), 
                        replace = TRUE)
  indices <- 
    as.vector(sapply(blockstarts, function(s)
                     (s + seq_len(blocklength) - 1) %% nobs + 1))[seq_len(nobs)]
  attr(indices, "starts") <- blockstarts
  attr(indices, "lengths") <- blocklength
  indices
\}

\nwlinkedidentc{bootindex_stationary}{NW29Cep0-B4OnS-1} <- function(nobs, blocklength) \{
  blockstarts <- sample(seq_len(nobs), replace = TRUE) 
  blocklengths <- rpois(nobs, blocklength)
  while(sum(blocklengths) < nobs) \{
    blockstarts <- c(blockstarts, sample(seq_len(nobs), replace = TRUE))
    blocklengths <- c(blocklengths, rpois(nobs, blocklength))
  \}
  fullblocks <- sum(cumsum(blocklengths) < nobs)
  blockstarts <- blockstarts[seq_len(fullblocks + 1)]
  blocklengths <- c(blocklengths[seq_len(fullblocks)],
                    nobs - sum(blocklengths[seq_len(fullblocks)]))
  indices <- unlist(sapply(seq_len(fullblocks + 1), function(s)
                           1 + seq.int(from = blockstarts[s] - 1, 
                                       length = blocklengths[s]) %% nobs))
  
  attr(indices, "starts") <- blockstarts[blocklengths != 0]
  attr(indices, "lengths") <- blocklengths[blocklengths != 0]
  indices
\}
\nwindexdefn{\nwixident{bootindex{\_}movingblock}}{bootindex:unmovingblock}{NW29Cep0-B4OnS-1}\nwindexdefn{\nwixident{bootindex{\_}circularblock}}{bootindex:uncircularblock}{NW29Cep0-B4OnS-1}\nwindexdefn{\nwixident{bootindex{\_}stationary}}{bootindex:unstationary}{NW29Cep0-B4OnS-1}\eatline
\nwnotused{bootindex.R}\nwidentdefs{\\{{\nwixident{bootindex{\_}circularblock}}{bootindex:uncircularblock}}\\{{\nwixident{bootindex{\_}movingblock}}{bootindex:unmovingblock}}\\{{\nwixident{bootindex{\_}stationary}}{bootindex:unstationary}}}\nwendcode{}\nwbegindocs{26}\nwdocspar
{\Tt{}\nwlinkedidentq{rbootindex}{NW29Cep0-3SDs62-1}\nwendquote} just chooses between the two boostrap functions
depending on the argument passed to {\Tt{}\nwlinkedidentq{mixedbootstrap}{NW29Cep0-3NQVMk-6}\nwendquote}.
\nwenddocs{}\nwbegincode{27}\sublabel{NW29Cep0-3SDs62-1}\nwmargintag{{\nwtagstyle{}\subpageref{NW29Cep0-3SDs62-1}}}\moddef{Define block bootstrap function \code{}rbootindex\edoc{}~{\nwtagstyle{}\subpageref{NW29Cep0-3SDs62-1}}}\endmoddef\nwstartdeflinemarkup\nwusesondefline{\\{NW29Cep0-3NQVMk-6}}\nwenddeflinemarkup
\nwlinkedidentc{rbootindex}{NW29Cep0-3SDs62-1} <- switch(bootstrap,
  moving = function() \nwlinkedidentc{bootindex_movingblock}{NW29Cep0-B4OnS-1}(noos, blocklength),
  circular = function() \nwlinkedidentc{bootindex_circularblock}{NW29Cep0-B4OnS-1}(noos, blocklength),
  stationary = function() \nwlinkedidentc{bootindex_stationary}{NW29Cep0-B4OnS-1}(noos, blocklength))
\nwindexdefn{\nwixident{rbootindex}}{rbootindex}{NW29Cep0-3SDs62-1}\eatline
\nwused{\\{NW29Cep0-3NQVMk-6}}\nwidentdefs{\\{{\nwixident{rbootindex}}{rbootindex}}}\nwidentuses{\\{{\nwixident{bootindex{\_}circularblock}}{bootindex:uncircularblock}}\\{{\nwixident{bootindex{\_}movingblock}}{bootindex:unmovingblock}}\\{{\nwixident{bootindex{\_}stationary}}{bootindex:unstationary}}}\nwindexuse{\nwixident{bootindex{\_}circularblock}}{bootindex:uncircularblock}{NW29Cep0-3SDs62-1}\nwindexuse{\nwixident{bootindex{\_}movingblock}}{bootindex:unmovingblock}{NW29Cep0-3SDs62-1}\nwindexuse{\nwixident{bootindex{\_}stationary}}{bootindex:unstationary}{NW29Cep0-3SDs62-1}\nwendcode{}\nwbegindocs{28}\nwdocspar

\section{Other functions}
\subsection{StepM (Romano and Wolf, 2005)} 

{\Tt{}\nwlinkedidentq{stepm}{NW29Cep0-HV7Dg-1}\nwendquote} implements \citepos{RoW:05} stepdown procedure to test
multiple models and allows for asymmetric two-sided tests as in
\citet{Cal:11e}.
\nwenddocs{}\nwbegincode{29}\sublabel{NW29Cep0-HV7Dg-1}\nwmargintag{{\nwtagstyle{}\subpageref{NW29Cep0-HV7Dg-1}}}\moddef{stepm.R~{\nwtagstyle{}\subpageref{NW29Cep0-HV7Dg-1}}}\endmoddef\nwstartdeflinemarkup\nwenddeflinemarkup
\LA{}Copyright and licensing~{\nwtagstyle{}\subpageref{NW29Cep0-4UiXc2-1}}\RA{}
\nwlinkedidentc{stepm}{NW29Cep0-HV7Dg-1} <- function(teststatistics, bootmatrix, lefttail, righttail) \{
  nstatistics <- length(teststatistics)
  rejected <- rep(FALSE, nstatistics)
  nrejected.endofloop <- 0
  repeat \{
    nrejected.topofloop <- nrejected.endofloop
    \LA{}Find critical values and determine which hypotheses are rejected~{\nwtagstyle{}\subpageref{NW29Cep0-4KyWj2-1}}\RA{}
    nrejected.endofloop <- sum(rejected)
    if (nrejected.endofloop == nrejected.topofloop ||
        nrejected.endofloop >= nstatistics) break
  \}
  return(list(leftcrit = leftcrit, rightcrit = rightcrit, 
              rejected = rejected))
\}
\nwindexdefn{\nwixident{stepm}}{stepm}{NW29Cep0-HV7Dg-1}\eatline
\nwnotused{stepm.R}\nwidentdefs{\\{{\nwixident{stepm}}{stepm}}}\nwendcode{}\nwbegindocs{30}\nwdocspar
The code determines which critical values to find (left or right) and
then which hypotheses we can reject based on these critical values.
Setting a tail to {\Tt{}NA\nwendquote} indicates that we're doing one-sided
alternatives.
\nwenddocs{}\nwbegincode{31}\sublabel{NW29Cep0-4KyWj2-1}\nwmargintag{{\nwtagstyle{}\subpageref{NW29Cep0-4KyWj2-1}}}\moddef{Find critical values and determine which hypotheses are rejected~{\nwtagstyle{}\subpageref{NW29Cep0-4KyWj2-1}}}\endmoddef\nwstartdeflinemarkup\nwusesondefline{\\{NW29Cep0-HV7Dg-1}}\nwenddeflinemarkup
bootmax <- apply(bootmatrix[!rejected,, drop = FALSE], 2, max)
bootmin <- apply(bootmatrix[!rejected,, drop = FALSE], 2, min)
rightcrit <- if (is.na(righttail)) Inf else 
               quantile(bootmax, 1 - righttail)
leftcrit <- if (is.na(lefttail)) -Inf else 
              quantile(bootmin[bootmax <= rightcrit], 
                       lefttail / (1 - righttail))
rejected[teststatistics < leftcrit | teststatistics > rightcrit] <- TRUE
\nwused{\\{NW29Cep0-HV7Dg-1}}\nwendcode{}\nwbegindocs{32}\nwdocspar

\subsection{Forecast estimation}

The {\Tt{}\nwlinkedidentq{recursive{\_}forecasts}{NW29Cep0-15rBrk-1}\nwendquote} function constructs a sequence of rolling,
recursive, or fixed window forecasts using {\Tt{}model\nwendquote}.  {\Tt{}model\nwendquote} must
be a function that takes subsets of {\Tt{}dataset\nwendquote} as an argument and
returns an object with a {\Tt{}predict\nwendquote} method.
\nwenddocs{}\nwbegincode{33}\sublabel{NW29Cep0-15rBrk-1}\nwmargintag{{\nwtagstyle{}\subpageref{NW29Cep0-15rBrk-1}}}\moddef{recursive-forecasts.R~{\nwtagstyle{}\subpageref{NW29Cep0-15rBrk-1}}}\endmoddef\nwstartdeflinemarkup\nwenddeflinemarkup
\LA{}Copyright and licensing~{\nwtagstyle{}\subpageref{NW29Cep0-4UiXc2-1}}\RA{}
\nwlinkedidentc{recursive_forecasts}{NW29Cep0-15rBrk-1} <- function(model, dataset, R,
                         window = c("recursive", "rolling", "fixed"),...) \{
  window <- match.arg(window)
  n <- nrow(dataset)
  if (R >= n) stop("dataset must have more than R observations")
  getprediction <- function(firstobs, lastobs, horizon)
    predict(model(dataset[seq(firstobs, lastobs, by = 1),],...),
            newdata = dataset[lastobs + horizon,])
  switch(window,
         recursive = sapply((R+1):n, function(s) getprediction(1, s-1, 1)),
         rolling   = sapply((R+1):n, function(s) getprediction(s-R, s-1, 1)),
         fixed     = getprediction(1, R, 1:(n-R)))
\}                 
\nwindexdefn{\nwixident{recursive{\_}forecasts}}{recursive:unforecasts}{NW29Cep0-15rBrk-1}\eatline
\nwnotused{recursive-forecasts.R}\nwidentdefs{\\{{\nwixident{recursive{\_}forecasts}}{recursive:unforecasts}}}\nwendcode{}\nwbegindocs{34}\nwdocspar
\subsection{Model manipulation}

{\Tt{}\nwlinkedidentq{extract{\_}target}{NW29Cep0-F2wuL-1}\nwendquote} and {\Tt{}\nwlinkedidentq{extract{\_}predictors}{NW29Cep0-F2wuL-1}\nwendquote} are designed to take the
{\Tt{}model\nwendquote} and {\Tt{}dataset\nwendquote} arguments that are passed to functions like
{\Tt{}\nwlinkedidentq{mixedwindow}{NW29Cep0-3NQVMk-4}\nwendquote} and return either the predictand or the predictors
for the {\Tt{}null\nwendquote} model.  These quantities are then used to approximate
the distribution of the \oos\ average.
\nwenddocs{}\nwbegincode{35}\sublabel{NW29Cep0-F2wuL-1}\nwmargintag{{\nwtagstyle{}\subpageref{NW29Cep0-F2wuL-1}}}\moddef{extract.R~{\nwtagstyle{}\subpageref{NW29Cep0-F2wuL-1}}}\endmoddef\nwstartdeflinemarkup\nwenddeflinemarkup
\LA{}Copyright and licensing~{\nwtagstyle{}\subpageref{NW29Cep0-4UiXc2-1}}\RA{}
\nwlinkedidentc{extract_predictors}{NW29Cep0-F2wuL-1} <- function(model, dataset)
  model.matrix(terms(model(dataset[1,])), data = dataset)

\nwlinkedidentc{extract_target}{NW29Cep0-F2wuL-1} <- function(model, dataset) \{
  target <- model.response(model.frame(terms(model(dataset[1,])), dataset))
  if (is.ts(dataset))
    target <- ts(target, start = start(dataset), frequency = frequency(dataset))
  target
\}
\nwindexdefn{\nwixident{extract{\_}predictors}}{extract:unpredictors}{NW29Cep0-F2wuL-1}\nwindexdefn{\nwixident{extract{\_}target}}{extract:untarget}{NW29Cep0-F2wuL-1}\eatline
\nwnotused{extract.R}\nwidentdefs{\\{{\nwixident{extract{\_}predictors}}{extract:unpredictors}}\\{{\nwixident{extract{\_}target}}{extract:untarget}}}\nwendcode{}\nwbegindocs{36}\nwdocspar
\subsection{Data generation}
I'm also going to define a simple function to generate a new dataset
from a given VAR model.  The {\Tt{}coefficients\nwendquote} argument is a list that
contains the vector of true coefficient values for each equation and
the innovations are assumed to be multivariate normal.  I'm going to
return a data frame with all of the lags listed explicitly, because
that will make the Monte Carlo easier.
\nwenddocs{}\nwbegincode{37}\sublabel{NW29Cep0-2Jbam7-1}\nwmargintag{{\nwtagstyle{}\subpageref{NW29Cep0-2Jbam7-1}}}\moddef{rvar.R~{\nwtagstyle{}\subpageref{NW29Cep0-2Jbam7-1}}}\endmoddef\nwstartdeflinemarkup\nwenddeflinemarkup
\nwlinkedidentc{rvar}{NW29Cep0-2Jbam7-1} <- function(nobs, coefficients, intercept, vcv, 
                 nburn = 1000, y0 = rep(0, nterms)) \{
  \LA{}Check and format \code{}rvar\edoc{} arguments~{\nwtagstyle{}\subpageref{NW29Cep0-KDAQP-1}}\RA{}
  y <- rbind(matrix(y0, nlag)[nlag:1,], mvrnorm(nobs + nburn, rep(0, neq), vcv))
  for (i in 1:(nobs + nburn)) \{
    xvec <- as.numeric(y[i + ((nlag-1):0),])
    y[i + nlag,] <-intercept + sapply(coefficients, function(b) 
                                      crossprod(xvec, b)) + y[i + nlag,]
  \}
  return(\LA{}data frame of simulated series~{\nwtagstyle{}\subpageref{NW29Cep0-3E2M0d-1}}\RA{})
\}
\nwindexdefn{\nwixident{rvar}}{rvar}{NW29Cep0-2Jbam7-1}\eatline
\nwnotused{rvar.R}\nwidentdefs{\\{{\nwixident{rvar}}{rvar}}}\nwendcode{}\nwbegindocs{38}We just do a few consistency checks on the arguments.  Also, if the
list of coefficients has names associated with each element, we take
those to be the names of the series; otherwise we assign generic names.
\nwenddocs{}\nwbegincode{39}\sublabel{NW29Cep0-KDAQP-1}\nwmargintag{{\nwtagstyle{}\subpageref{NW29Cep0-KDAQP-1}}}\moddef{Check and format \code{}rvar\edoc{} arguments~{\nwtagstyle{}\subpageref{NW29Cep0-KDAQP-1}}}\endmoddef\nwstartdeflinemarkup\nwusesondefline{\\{NW29Cep0-2Jbam7-1}}\nwenddeflinemarkup
nterms <- length(coefficients[[1]])
if (any(sapply(coefficients[-1], length) != nterms))
  stop("All of the elememts of 'coefficients' must have the same length")
neq <- length(coefficients)
if (is.null(names(coefficients))) \{
  eqnames <- paste("x", seq_len(neq), sep = "") 
\} else \{
  eqnames <- names(coefficients)
\}
nlag <- nterms %/% neq
if (!isTRUE(all.equal(nlag * neq, nterms))) 
  stop("The number of lags must be an integer")
y0 <- as.numeric(y0)
\nwused{\\{NW29Cep0-2Jbam7-1}}\nwendcode{}\nwbegindocs{40}\nwdocspar

The remaining code in this function just reorders the generated
series, assigns names, and puts them in a data frame that will be
convenient for regression estimation.
\nwenddocs{}\nwbegincode{41}\sublabel{NW29Cep0-3E2M0d-1}\nwmargintag{{\nwtagstyle{}\subpageref{NW29Cep0-3E2M0d-1}}}\moddef{data frame of simulated series~{\nwtagstyle{}\subpageref{NW29Cep0-3E2M0d-1}}}\endmoddef\nwstartdeflinemarkup\nwusesondefline{\\{NW29Cep0-2Jbam7-1}}\nwenddeflinemarkup
as.data.frame(do.call("cbind", lapply(1:neq, function(j) \{
    jmatrix <- sapply((nlag+1):1, function(i) y[(1:nobs) + i + nburn - 1, j])
    colnames(jmatrix) <- c(eqnames[j], paste(eqnames[j], 1:nlag, sep = "L"))
    jmatrix
  \})))
\nwused{\\{NW29Cep0-2Jbam7-1}}\nwendcode{}\nwbegindocs{42}\nwdocspar

\section{Objects, classes, and methods}

Some of these functions work through the {\Tt{}predict\nwendquote} method.  This is
great if we want to use, for example, \textsc{ols} forecasts; but in
the empirical section of the paper we also use some nonlinear models.
So we'll define {\Tt{}predict\nwendquote} methods for those models as well.

The first is a {\Tt{}predict\nwendquote} method for \citepos{CaT:08} modification of
\citepos{GoW:08} \textsc{ols} forecasts: \citet{CaT:08} impose that
the forecast is nonnegative, arguing that it is implausible for the
equity premium to be negative.  I'll first make a simple constructor
function for the {\Tt{}\nwlinkedidentq{CT}{NW29Cep0-4VkTYt-1}\nwendquote} class.  There's some very minor error
checking; just that a {\Tt{}predict\nwendquote} method exists for at least one of
the classes of the argument, {\Tt{}model\nwendquote}.
\nwenddocs{}\nwbegincode{43}\sublabel{NW29Cep0-4VkTYt-1}\nwmargintag{{\nwtagstyle{}\subpageref{NW29Cep0-4VkTYt-1}}}\moddef{Classes.R~{\nwtagstyle{}\subpageref{NW29Cep0-4VkTYt-1}}}\endmoddef\nwstartdeflinemarkup\nwprevnextdefs{\relax}{NW29Cep0-4VkTYt-2}\nwenddeflinemarkup
\LA{}Copyright and licensing~{\nwtagstyle{}\subpageref{NW29Cep0-4UiXc2-1}}\RA{}
\nwlinkedidentc{CT}{NW29Cep0-4VkTYt-1} <- function(model) \{
  \LA{}Check inputs for \code{}CT\edoc{}~{\nwtagstyle{}\subpageref{NW29Cep0-12DbnH-1}}\RA{}
  class(model) <- c("\nwlinkedidentc{CT}{NW29Cep0-4VkTYt-1}", class(model))
  model
\}
\nwindexdefn{\nwixident{CT}}{CT}{NW29Cep0-4VkTYt-1}\eatline
\nwalsodefined{\\{NW29Cep0-4VkTYt-2}\\{NW29Cep0-4VkTYt-3}\\{NW29Cep0-4VkTYt-4}\\{NW29Cep0-4VkTYt-5}}\nwnotused{Classes.R}\nwidentdefs{\\{{\nwixident{CT}}{CT}}}\nwendcode{}\nwbegindocs{44}\nwdocspar
\nwenddocs{}\nwbegincode{45}\sublabel{NW29Cep0-12DbnH-1}\nwmargintag{{\nwtagstyle{}\subpageref{NW29Cep0-12DbnH-1}}}\moddef{Check inputs for \code{}CT\edoc{}~{\nwtagstyle{}\subpageref{NW29Cep0-12DbnH-1}}}\endmoddef\nwstartdeflinemarkup\nwusesondefline{\\{NW29Cep0-4VkTYt-1}}\nwenddeflinemarkup
  if (!\nwlinkedidentc{HasMethod}{NW29Cep0-4VkTYt-5}(model, "predict"))
    stop("'model' must have a 'predict' method.")
\nwused{\\{NW29Cep0-4VkTYt-1}}\nwidentuses{\\{{\nwixident{HasMethod}}{HasMethod}}}\nwindexuse{\nwixident{HasMethod}}{HasMethod}{NW29Cep0-12DbnH-1}\nwendcode{}\nwbegindocs{46}\nwdocspar

The prediction method is pretty straightforward: use the {\Tt{}predictm\nwendquote}
method for the original model, 
but return zero if the forecast is negative.
\nwenddocs{}\nwbegincode{47}\sublabel{NW29Cep0-4VkTYt-2}\nwmargintag{{\nwtagstyle{}\subpageref{NW29Cep0-4VkTYt-2}}}\moddef{Classes.R~{\nwtagstyle{}\subpageref{NW29Cep0-4VkTYt-1}}}\plusendmoddef\nwstartdeflinemarkup\nwprevnextdefs{NW29Cep0-4VkTYt-1}{NW29Cep0-4VkTYt-3}\nwenddeflinemarkup
predict.\nwlinkedidentc{CT}{NW29Cep0-4VkTYt-1}\nwlinkedidentc{ <- function(object, newdata,...) \{}{NW29Cep0-4VkTYt-2}
  predictions <- NextMethod(object, newdata,...)
  ifelse(predictions > 0, predictions, 0)
\}
\nwindexdefn{\nwixident{predict.CT}}{predict.CT}{NW29Cep0-4VkTYt-2}\eatline
\nwidentdefs{\\{{\nwixident{predict.CT}}{predict.CT}}}\nwidentuses{\\{{\nwixident{CT}}{CT}}}\nwindexuse{\nwixident{CT}}{CT}{NW29Cep0-4VkTYt-2}\nwendcode{}\nwbegindocs{48}If I wanted to develop this object further it would make sense to add
print, summary, and plot methods (at least), but I'm not going to
worry about that for now.

I also want to make some objects that will represent aggregate
forecasts, the average or median of several models, for example.  I'll
do this by adding an {\Tt{}\nwlinkedidentq{Aggregate}{NW29Cep0-4VkTYt-3}\nwendquote} class to a list of models.  Again,
the constructor comes first.
\nwenddocs{}\nwbegincode{49}\sublabel{NW29Cep0-4VkTYt-3}\nwmargintag{{\nwtagstyle{}\subpageref{NW29Cep0-4VkTYt-3}}}\moddef{Classes.R~{\nwtagstyle{}\subpageref{NW29Cep0-4VkTYt-1}}}\plusendmoddef\nwstartdeflinemarkup\nwprevnextdefs{NW29Cep0-4VkTYt-2}{NW29Cep0-4VkTYt-4}\nwenddeflinemarkup
\nwlinkedidentc{Aggregate}{NW29Cep0-4VkTYt-3} <- function(model.list, fn) \{
  \LA{}Check inputs to \code{}Aggregate\edoc{}~{\nwtagstyle{}\subpageref{NW29Cep0-4U3W6i-1}}\RA{}
  class(model.list) <- "\nwlinkedidentc{Aggregate}{NW29Cep0-4VkTYt-3}"
  attr(model.list, "Aggregator") <- fn
  model.list
\}
\nwindexdefn{\nwixident{Aggregate}}{Aggregate}{NW29Cep0-4VkTYt-3}\eatline
\nwidentdefs{\\{{\nwixident{Aggregate}}{Aggregate}}}\nwendcode{}\nwbegindocs{50}Checking the inputs for {\Tt{}\nwlinkedidentq{Aggregate}{NW29Cep0-4VkTYt-3}\nwendquote} is pretty modest for now,
although I may add more checks if they turn out to be important.
\nwenddocs{}\nwbegincode{51}\sublabel{NW29Cep0-4U3W6i-1}\nwmargintag{{\nwtagstyle{}\subpageref{NW29Cep0-4U3W6i-1}}}\moddef{Check inputs to \code{}Aggregate\edoc{}~{\nwtagstyle{}\subpageref{NW29Cep0-4U3W6i-1}}}\endmoddef\nwstartdeflinemarkup\nwusesondefline{\\{NW29Cep0-4VkTYt-3}}\nwenddeflinemarkup
if (!is.function(fn)) stop("'fn' must be a function.")
if (!all(sapply(model.list, function(m) \nwlinkedidentc{HasMethod}{NW29Cep0-4VkTYt-5}(m, "predict"))))
  stop("Each element of 'model.list' must have a 'predict' method defined.")
\nwused{\\{NW29Cep0-4VkTYt-3}}\nwidentuses{\\{{\nwixident{HasMethod}}{HasMethod}}}\nwindexuse{\nwixident{HasMethod}}{HasMethod}{NW29Cep0-4U3W6i-1}\nwendcode{}\nwbegindocs{52}\nwdocspar

The {\Tt{}predict\nwendquote} method now just consists of calling {\Tt{}predict\nwendquote}
element-by-element and then combining the forecasts through the
predefined function.
\nwenddocs{}\nwbegincode{53}\sublabel{NW29Cep0-4VkTYt-4}\nwmargintag{{\nwtagstyle{}\subpageref{NW29Cep0-4VkTYt-4}}}\moddef{Classes.R~{\nwtagstyle{}\subpageref{NW29Cep0-4VkTYt-1}}}\plusendmoddef\nwstartdeflinemarkup\nwprevnextdefs{NW29Cep0-4VkTYt-3}{NW29Cep0-4VkTYt-5}\nwenddeflinemarkup
predict.\nwlinkedidentc{Aggregate}{NW29Cep0-4VkTYt-3}\nwlinkedidentc{ <- function(object, newdata,...) \{}{NW29Cep0-4VkTYt-4}
  \LA{}Forecast with each individual model~{\nwtagstyle{}\subpageref{NW29Cep0-1jm0YB-1}}\RA{}
  if (nrow(newdata) > 1) \{
    \LA{}Return multiple-horizon aggregate forecasts~{\nwtagstyle{}\subpageref{NW29Cep0-2Zg6lR-1}}\RA{}
  \} else \{
    \LA{}Return aggregate single period forecast~{\nwtagstyle{}\subpageref{NW29Cep0-zSOm4-1}}\RA{}
  \}
\}
\nwindexdefn{\nwixident{predict.Aggregate}}{predict.Aggregate}{NW29Cep0-4VkTYt-4}\eatline
\nwidentdefs{\\{{\nwixident{predict.Aggregate}}{predict.Aggregate}}}\nwidentuses{\\{{\nwixident{Aggregate}}{Aggregate}}}\nwindexuse{\nwixident{Aggregate}}{Aggregate}{NW29Cep0-4VkTYt-4}\nwendcode{}\nwbegindocs{54}\nwdocspar
\nwenddocs{}\nwbegincode{55}\sublabel{NW29Cep0-1jm0YB-1}\nwmargintag{{\nwtagstyle{}\subpageref{NW29Cep0-1jm0YB-1}}}\moddef{Forecast with each individual model~{\nwtagstyle{}\subpageref{NW29Cep0-1jm0YB-1}}}\endmoddef\nwstartdeflinemarkup\nwusesondefline{\\{NW29Cep0-4VkTYt-4}}\nwenddeflinemarkup
arguments <- list(...)
arguments$newdata <- newdata
forecasts <- sapply(object, function(model) 
                    do.call("predict", c(list(object = model), arguments)))
\nwused{\\{NW29Cep0-4VkTYt-4}}\nwendcode{}\nwbegindocs{56}\nwdocspar

\nwenddocs{}\nwbegincode{57}\sublabel{NW29Cep0-2Zg6lR-1}\nwmargintag{{\nwtagstyle{}\subpageref{NW29Cep0-2Zg6lR-1}}}\moddef{Return multiple-horizon aggregate forecasts~{\nwtagstyle{}\subpageref{NW29Cep0-2Zg6lR-1}}}\endmoddef\nwstartdeflinemarkup\nwusesondefline{\\{NW29Cep0-4VkTYt-4}}\nwenddeflinemarkup
return(apply(forecasts, 1, attr(object, "Aggregator")))
\nwused{\\{NW29Cep0-4VkTYt-4}}\nwendcode{}\nwbegindocs{58}\nwdocspar

\nwenddocs{}\nwbegincode{59}\sublabel{NW29Cep0-zSOm4-1}\nwmargintag{{\nwtagstyle{}\subpageref{NW29Cep0-zSOm4-1}}}\moddef{Return aggregate single period forecast~{\nwtagstyle{}\subpageref{NW29Cep0-zSOm4-1}}}\endmoddef\nwstartdeflinemarkup\nwusesondefline{\\{NW29Cep0-4VkTYt-4}}\nwenddeflinemarkup
return(attr(object, "Aggregator")(forecasts))
\nwused{\\{NW29Cep0-4VkTYt-4}}\nwendcode{}\nwbegindocs{60}\nwdocspar

Finally, let's define a fairly simple function that checks whether a
given function exists for a given S3 class.  I'm sure that such a
function exists already, but I don't know what it's called.  This
function is used above to check that different arguments have a
{\Tt{}predict\nwendquote} method defined, but I guess it could be used more generally.
\nwenddocs{}\nwbegincode{61}\sublabel{NW29Cep0-4VkTYt-5}\nwmargintag{{\nwtagstyle{}\subpageref{NW29Cep0-4VkTYt-5}}}\moddef{Classes.R~{\nwtagstyle{}\subpageref{NW29Cep0-4VkTYt-1}}}\plusendmoddef\nwstartdeflinemarkup\nwprevnextdefs{NW29Cep0-4VkTYt-4}{\relax}\nwenddeflinemarkup
\nwlinkedidentc{HasMethod}{NW29Cep0-4VkTYt-5} <- function(object, method.name) \{
  sapply(method.name, function(name) \{
    listed.methods <- methods(name)
    any(sapply(class(object), function(eachclass)
               length(grep(eachclass, listed.methods))) > 0)
  \})
\}
\nwindexdefn{\nwixident{HasMethod}}{HasMethod}{NW29Cep0-4VkTYt-5}\eatline
\nwidentdefs{\\{{\nwixident{HasMethod}}{HasMethod}}}\nwendcode{}\nwbegindocs{62}\nwdocspar
\appendix

\section{Lookup tables for \citet{Mcc:07} critical values}

\nwenddocs{}\nwbegincode{63}\sublabel{NW29Cep0-18ASnA-1}\nwmargintag{{\nwtagstyle{}\subpageref{NW29Cep0-18ASnA-1}}}\moddef{mccrackendata.R~{\nwtagstyle{}\subpageref{NW29Cep0-18ASnA-1}}}\endmoddef\nwstartdeflinemarkup\nwenddeflinemarkup
\nwlinkedidentc{PiIntervals}{NW29Cep0-18ASnA-1} <- c(0, .1, seq(from = 0.2, to = 2.0, by = .2))
\nwlinkedidentc{PiBounds}{NW29Cep0-18ASnA-1} <- range(\nwlinkedidentc{PiIntervals}{NW29Cep0-18ASnA-1})
\nwlinkedidentc{McCrackenData}{NW29Cep0-18ASnA-1} <- list()
\nwlinkedidentc{McCrackenData}{NW29Cep0-18ASnA-1}[["rolling"]] <- array(dim = c(12, 3, 10), dimnames = c("pi", "confidence", "k2"), 
  data = 
  c(2.326,1.875,1.799,1.604, 1.447, 1.340, 1.221, 1.179, 1.098, 1.021, 0.969, 0.882,
    1.645,1.251,1.117,0.970, 0.859, 0.722, 0.651, 0.575, 0.510, 0.455, 0.382, 0.334,
    1.280,0.903,0.776,0.637, 0.530, 0.401, 0.317, 0.246, 0.180, 0.136, 0.116, 0.078,
    2.326,1.959,1.757,1.504, 1.325, 1.180, 1.165, 0.996, 0.953, 0.883, 0.744, 0.640,
    1.645,1.280,1.105,0.884, 0.753, 0.631, 0.484, 0.401, 0.304, 0.235, 0.166, 0.103,
    1.280,0.915,0.755,0.569, 0.425, 0.280, 0.155, 0.111, 0.026,-0.050,-0.094,-0.140,
    2.326,1.860,1.669,1.473, 1.271, 1.076, 0.984, 0.896, 0.773, 0.614, 0.504, 0.431,
    1.645,1.274,1.088,0.842, 0.667, 0.490, 0.381, 0.251, 0.146, 0.066,-0.016,-0.084,
    1.280,0.938,0.718,0.521, 0.346, 0.201, 0.064,-0.042,-0.137,-0.224,-0.302,-0.346,
    2.326,1.905,1.700,1.503, 1.183, 1.003, 0.903, 0.755, 0.656, 0.455, 0.342, 0.234,
    1.645,1.267,1.087,0.852, 0.585, 0.376, 0.274, 0.136, 0.024,-0.080,-0.173,-0.222,
    1.280,0.866,0.731,0.494, 0.248, 0.098,-0.047,-0.164,-0.262,-0.362,-0.434,-0.505,
    2.326,1.881,1.627,1.347, 1.112, 0.927, 0.790, 0.657, 0.504, 0.307, 0.193, 0.123,
    1.645,1.229,1.034,0.716, 0.479, 0.280, 0.155,-0.019,-0.090,-0.219,-0.329,-0.385,
    1.280,0.825,0.694,0.402, 0.154,-0.025,-0.168,-0.305,-0.399,-0.508,-0.589,-0.674,
    2.326,1.826,1.680,1.312, 1.007, 0.850, 0.641, 0.558, 0.336, 0.195, 0.069, 0.017,
    1.645,1.176,0.966,0.621, 0.407, 0.225, 0.058,-0.119,-0.218,-0.336,-0.428,-0.535,
    1.280,0.811,0.602,0.319, 0.088,-0.095,-0.262,-0.423,-0.523,-0.638,-0.732,-0.821,
    2.326,1.842,1.620,1.233, 0.989, 0.751, 0.526, 0.485, 0.227, 0.055,-0.039,-0.127,
    1.645,1.154,0.936,0.628, 0.346, 0.171,-0.011,-0.182,-0.320,-0.433,-0.531,-0.663,
    1.280,0.791,0.573,0.279, 0.038,-0.157,-0.326,-0.497,-0.611,-0.750,-0.841,-0.933,
    2.326,1.819,1.582,1.178, 0.918, 0.702, 0.466, 0.349, 0.132,-0.018,-0.176,-0.302,
    1.645,1.157,0.924,0.562, 0.258, 0.081,-0.099,-0.281,-0.432,-0.552,-0.672,-0.785,
    1.280,0.758,0.541,0.244,-0.042,-0.244,-0.408,-0.576,-0.727,-0.838,-0.957,-1.040,
    2.326,1.768,1.510,1.110, 0.845, 0.600, 0.408, 0.235, 0.036,-0.099,-0.277,-0.407,
    1.645,1.117,0.892,0.504, 0.213, 0.021,-0.156,-0.374,-0.529,-0.623,-0.785,-0.885,
    1.280,0.742,0.520,0.193,-0.105,-0.322,-0.491,-0.657,-0.803,-0.951,-1.049,-1.153,
    2.326,1.713,1.428,1.075, 0.808, 0.536, 0.298, 0.122,-0.064,-0.248,-0.381,-0.482,
    1.645,1.068,0.872,0.443, 0.133,-0.038,-0.258,-0.466,-0.605,-0.765,-0.909,-1.011,
    1.280,0.727,0.500,0.138,-0.144,-0.374,-0.568,-0.757,-0.902,-1.045,-1.167,-1.288))

\nwlinkedidentc{McCrackenData}{NW29Cep0-18ASnA-1}[["recursive"]] <- array(dim = c(12, 3, 10), dimnames = c("pi", "confidence", "k2"), 
  data =
  c(2.326,1.921,1.784,1.625,1.515, 1.462, 1.436, 1.413, 1.343, 1.316, 1.274, 1.238,
    1.645,1.245,1.111,0.994,0.971, 0.863, 0.771, 0.740, 0.705, 0.671, 0.638, 0.610,
    1.280,0.885,0.780,0.657,0.598, 0.512, 0.443, 0.402, 0.370, 0.330, 0.306, 0.281,
    2.326,1.986,1.856,1.563,1.436, 1.387, 1.312, 1.276, 1.196, 1.158, 1.127, 1.074,
    1.645,1.274,1.140,0.986,0.868, 0.782, 0.704, 0.623, 0.596, 0.537, 0.507, 0.478,
    1.280,0.932,0.786,0.614,0.541, 0.455, 0.361, 0.295, 0.253, 0.235, 0.194, 0.160,
    2.326,1.840,1.737,1.542,1.448, 1.359, 1.252, 1.148, 1.071, 0.976, 0.978, 0.953,
    1.645,1.300,1.120,0.968,0.808, 0.685, 0.610, 0.552, 0.496, 0.438, 0.419, 0.386,
    1.280,0.939,0.751,0.551,0.454, 0.356, 0.279, 0.222, 0.175, 0.108, 0.074, 0.035,
    2.326,1.872,1.731,1.581,1.365, 1.195, 1.119, 1.108, 1.041, 0.902, 0.861, 0.854,
    1.645,1.264,1.101,0.914,0.772, 0.609, 0.502, 0.419, 0.345, 0.285, 0.239, 0.221,
    1.280,0.898,0.742,0.562,0.419, 0.263, 0.169, 0.094, 0.052,-0.014,-0.054,-0.106,
    2.326,1.849,1.679,1.468,1.242, 1.095, 0.995, 0.979, 0.913, 0.795, 0.732, 0.677,
    1.645,1.222,1.061,0.849,0.689, 0.491, 0.386, 0.308, 0.224, 0.148, 0.107, 0.081,
    1.280,0.866,0.694,0.461,0.315, 0.179, 0.062,-0.021,-0.083,-0.145,-0.174,-0.228,
    2.326,1.836,1.639,1.390,1.200, 1.042, 0.943, 0.859, 0.755, 0.686, 0.610, 0.593,
    1.645,1.192,0.998,0.768,0.615, 0.429, 0.328, 0.259, 0.141, 0.078, 0.055,-0.019,
    1.280,0.823,0.642,0.394,0.256, 0.108,-0.011,-0.101,-0.164,-0.218,-0.266,-0.319,
    2.326,1.836,1.649,1.341,1.154, 0.994, 0.872, 0.810, 0.637, 0.549, 0.476, 0.438,
    1.645,1.199,0.976,0.742,0.546, 0.372, 0.279, 0.191, 0.072,-0.002,-0.034,-0.105,
    1.280,0.811,0.615,0.359,0.213, 0.062,-0.088,-0.152,-0.230,-0.305,-0.363,-0.449,
    2.326,1.789,1.659,1.298,1.090, 0.879, 0.788, 0.728, 0.503, 0.444, 0.401, 0.359,
    1.645,1.193,0.928,0.677,0.462, 0.302, 0.198, 0.105, 0.020,-0.058,-0.101,-0.176,
    1.280,0.773,0.574,0.329,0.139, 0.003,-0.131,-0.203,-0.293,-0.383,-0.452,-0.516,
    2.326,1.813,1.607,1.268,1.112, 0.804, 0.724, 0.634, 0.523, 0.427, 0.391, 0.305,
    1.645,1.112,0.912,0.617,0.397, 0.276, 0.121, 0.030,-0.055,-0.122,-0.193,-0.257,
    1.280,0.733,0.561,0.273,0.096,-0.068,-0.187,-0.286,-0.377,-0.437,-0.518,-0.579,
    2.326,1.743,1.534,1.193,1.035, 0.758, 0.621, 0.506, 0.419, 0.347, 0.285, 0.185,
    1.645,1.082,0.890,0.566,0.358, 0.205, 0.043,-0.072,-0.162,-0.222,-0.296,-0.339,
    1.280,0.749,0.529,0.226,0.032,-0.130,-0.248,-0.355,-0.454,-0.524,-0.591,-0.651))

\nwlinkedidentc{McCrackenData}{NW29Cep0-18ASnA-1}[["fixed"]] <- array(dim = c(12, 3, 10), dimnames = c("pi", "confidence", "k2"), 
  data =
  c(2.326,2.201,2.051,1.974,2.061, 2.037, 2.024, 1.992, 2.018, 1.996, 2.016, 1.993,
    1.645,1.506,1.416,1.364,1.428, 1.346, 1.252, 1.301, 1.293, 1.249, 1.235, 1.218,
    1.280,1.149,1.079,1.042,1.040, 0.976, 0.917, 0.896, 0.893, 0.908, 0.834, 0.862,
    2.326,2.145,2.089,1.923,1.947, 1.964, 1.749, 1.751, 1.665, 1.725, 1.646, 1.613,
    1.645,1.468,1.342,1.301,1.265, 1.164, 1.072, 1.034, 1.046, 0.977, 0.982, 0.955,
    1.280,1.096,0.999,0.901,0.873, 0.798, 0.711, 0.680, 0.639, 0.578, 0.556, 0.520,
    2.326,2.045,1.977,1.957,1.805, 1.739, 1.602, 1.520, 1.597, 1.463, 1.513, 1.407,
    1.645,1.432,1.277,1.195,1.095, 1.014, 0.909, 0.893, 0.851, 0.761, 0.735, 0.733,
    1.280,1.063,0.922,0.793,0.705, 0.621, 0.540, 0.511, 0.455, 0.386, 0.373, 0.306,
    2.326,2.013,1.883,1.829,1.687, 1.528, 1.467, 1.475, 1.422, 1.318, 1.255, 1.277,
    1.645,1.369,1.281,1.110,0.997, 0.883, 0.755, 0.689, 0.650, 0.607, 0.566, 0.509,
    1.280,1.004,0.895,0.764,0.575, 0.476, 0.367, 0.340, 0.273, 0.204, 0.171, 0.081,
    2.326,1.930,1.878,1.716,1.596, 1.405, 1.254, 1.301, 1.230, 1.171, 1.115, 1.034,
    1.645,1.333,1.193,1.009,0.863, 0.725, 0.646, 0.570, 0.486, 0.410, 0.365, 0.291,
    1.280,0.945,0.838,0.636,0.487, 0.374, 0.258, 0.193, 0.115, 0.020,-0.022,-0.085,
    2.326,1.933,1.874,1.628,1.481, 1.382, 1.146, 1.188, 1.091, 1.016, 1.007, 0.878,
    1.645,1.269,1.122,0.936,0.771, 0.652, 0.538, 0.487, 0.367, 0.314, 0.222, 0.152,
    1.280,0.912,0.764,0.552,0.400, 0.299, 0.169, 0.103, 0.003,-0.106,-0.146,-0.235,
    2.326,1.925,1.859,1.556,1.377, 1.257, 1.105, 1.103, 0.987, 0.896, 0.828, 0.765,
    1.645,1.263,1.086,0.878,0.692, 0.557, 0.446, 0.346, 0.254, 0.191, 0.074, 0.014,
    1.280,0.895,0.731,0.513,0.332, 0.215, 0.060,-0.003,-0.147,-0.252,-0.308,-0.386,
    2.326,1.856,1.827,1.467,1.245, 1.146, 1.029, 0.980, 0.860, 0.786, 0.762, 0.666,
    1.645,1.249,1.064,0.807,0.623, 0.481, 0.363, 0.268, 0.151, 0.054,-0.042,-0.120,
    1.280,0.868,0.663,0.467,0.247, 0.153,-0.029,-0.115,-0.227,-0.343,-0.440,-0.502,
    2.326,1.878,1.697,1.440,1.198, 1.124, 0.902, 0.791, 0.683, 0.644, 0.595, 0.507,
    1.645,1.197,1.031,0.754,0.537, 0.416, 0.305, 0.162, 0.050,-0.067,-0.171,-0.242,
    1.280,0.844,0.655,0.396,0.182, 0.034,-0.111,-0.224,-0.303,-0.437,-0.543,-0.625,
    2.326,1.824,1.604,1.354,1.126, 0.998, 0.797, 0.659, 0.557, 0.550, 0.505, 0.415,
    1.645,1.143,1.007,0.688,0.455, 0.337, 0.167, 0.040,-0.057,-0.174,-0.246,-0.358,
    1.280,0.797,0.616,0.348,0.125,-0.055,-0.210,-0.305,-0.398,-0.559,-0.645,-0.729))
\nwindexdefn{\nwixident{McCrackenData}}{McCrackenData}{NW29Cep0-18ASnA-1}\nwindexdefn{\nwixident{PiBounds}}{PiBounds}{NW29Cep0-18ASnA-1}\nwindexdefn{\nwixident{PiIntervals}}{PiIntervals}{NW29Cep0-18ASnA-1}\eatline
\nwnotused{mccrackendata.R}\nwidentdefs{\\{{\nwixident{McCrackenData}}{McCrackenData}}\\{{\nwixident{PiBounds}}{PiBounds}}\\{{\nwixident{PiIntervals}}{PiIntervals}}}\nwendcode{}\nwbegindocs{64}\nwdocspar
\section{Namespace}
The {\Tt{}NAMESPACE\nwendquote} file specifies which of the functions defined in
this package are meant to be accessible to the users (among other things).

\nwenddocs{}\nwbegincode{65}\sublabel{NW29Cep0-15U98S-1}\nwmargintag{{\nwtagstyle{}\subpageref{NW29Cep0-15U98S-1}}}\moddef{NAMESPACE~{\nwtagstyle{}\subpageref{NW29Cep0-15U98S-1}}}\endmoddef\nwstartdeflinemarkup\nwenddeflinemarkup
\LA{}Copyright and licensing~{\nwtagstyle{}\subpageref{NW29Cep0-4UiXc2-1}}\RA{}
export(\nwlinkedidentc{clarkwest}{NW29Cep0-2ARW6h-1}, \nwlinkedidentc{mixedwindow}{NW29Cep0-3NQVMk-4}, \nwlinkedidentc{mixedbootstrap}{NW29Cep0-3NQVMk-6}, \nwlinkedidentc{recursive_forecasts}{NW29Cep0-15rBrk-1}, \nwlinkedidentc{stepm}{NW29Cep0-HV7Dg-1},
       \nwlinkedidentc{dmw_calculation}{NW29Cep0-3NQVMk-1}, \nwlinkedidentc{bootindex_circularblock}{NW29Cep0-B4OnS-1}, \nwlinkedidentc{bootindex_movingblock}{NW29Cep0-B4OnS-1},
       \nwlinkedidentc{extract_target}{NW29Cep0-F2wuL-1}, \nwlinkedidentc{extract_predictors}{NW29Cep0-F2wuL-1}, \nwlinkedidentc{CT}{NW29Cep0-4VkTYt-1}, \nwlinkedidentc{Aggregate}{NW29Cep0-4VkTYt-3}, \nwlinkedidentc{HasMethod}{NW29Cep0-4VkTYt-5},
       predict.\nwlinkedidentc{CT}{NW29Cep0-4VkTYt-1}\nwlinkedidentc{, predict.}{NW29Cep0-4VkTYt-2}\nwlinkedidentc{Aggregate}{NW29Cep0-4VkTYt-3}\nwlinkedidentc{, }{NW29Cep0-4VkTYt-4}\nwlinkedidentc{rvar}{NW29Cep0-2Jbam7-1}, \nwlinkedidentc{bootindex_stationary}{NW29Cep0-B4OnS-1}, 
       \nwlinkedidentc{mccracken_criticalvalue}{NW29Cep0-HXXXl-1}, \nwlinkedidentc{dmw_mse}{NW29Cep0-3NQVMk-3})
importFrom(MASS, mvrnorm)
\nwnotused{NAMESPACE}\nwidentuses{\\{{\nwixident{Aggregate}}{Aggregate}}\\{{\nwixident{bootindex{\_}circularblock}}{bootindex:uncircularblock}}\\{{\nwixident{bootindex{\_}movingblock}}{bootindex:unmovingblock}}\\{{\nwixident{bootindex{\_}stationary}}{bootindex:unstationary}}\\{{\nwixident{clarkwest}}{clarkwest}}\\{{\nwixident{CT}}{CT}}\\{{\nwixident{dmw{\_}calculation}}{dmw:uncalculation}}\\{{\nwixident{dmw{\_}mse}}{dmw:unmse}}\\{{\nwixident{extract{\_}predictors}}{extract:unpredictors}}\\{{\nwixident{extract{\_}target}}{extract:untarget}}\\{{\nwixident{HasMethod}}{HasMethod}}\\{{\nwixident{mccracken{\_}criticalvalue}}{mccracken:uncriticalvalue}}\\{{\nwixident{mixedbootstrap}}{mixedbootstrap}}\\{{\nwixident{mixedwindow}}{mixedwindow}}\\{{\nwixident{predict.Aggregate}}{predict.Aggregate}}\\{{\nwixident{predict.CT}}{predict.CT}}\\{{\nwixident{recursive{\_}forecasts}}{recursive:unforecasts}}\\{{\nwixident{rvar}}{rvar}}\\{{\nwixident{stepm}}{stepm}}}\nwindexuse{\nwixident{Aggregate}}{Aggregate}{NW29Cep0-15U98S-1}\nwindexuse{\nwixident{bootindex{\_}circularblock}}{bootindex:uncircularblock}{NW29Cep0-15U98S-1}\nwindexuse{\nwixident{bootindex{\_}movingblock}}{bootindex:unmovingblock}{NW29Cep0-15U98S-1}\nwindexuse{\nwixident{bootindex{\_}stationary}}{bootindex:unstationary}{NW29Cep0-15U98S-1}\nwindexuse{\nwixident{clarkwest}}{clarkwest}{NW29Cep0-15U98S-1}\nwindexuse{\nwixident{CT}}{CT}{NW29Cep0-15U98S-1}\nwindexuse{\nwixident{dmw{\_}calculation}}{dmw:uncalculation}{NW29Cep0-15U98S-1}\nwindexuse{\nwixident{dmw{\_}mse}}{dmw:unmse}{NW29Cep0-15U98S-1}\nwindexuse{\nwixident{extract{\_}predictors}}{extract:unpredictors}{NW29Cep0-15U98S-1}\nwindexuse{\nwixident{extract{\_}target}}{extract:untarget}{NW29Cep0-15U98S-1}\nwindexuse{\nwixident{HasMethod}}{HasMethod}{NW29Cep0-15U98S-1}\nwindexuse{\nwixident{mccracken{\_}criticalvalue}}{mccracken:uncriticalvalue}{NW29Cep0-15U98S-1}\nwindexuse{\nwixident{mixedbootstrap}}{mixedbootstrap}{NW29Cep0-15U98S-1}\nwindexuse{\nwixident{mixedwindow}}{mixedwindow}{NW29Cep0-15U98S-1}\nwindexuse{\nwixident{predict.Aggregate}}{predict.Aggregate}{NW29Cep0-15U98S-1}\nwindexuse{\nwixident{predict.CT}}{predict.CT}{NW29Cep0-15U98S-1}\nwindexuse{\nwixident{recursive{\_}forecasts}}{recursive:unforecasts}{NW29Cep0-15U98S-1}\nwindexuse{\nwixident{rvar}}{rvar}{NW29Cep0-15U98S-1}\nwindexuse{\nwixident{stepm}}{stepm}{NW29Cep0-15U98S-1}\nwendcode{}\nwbegindocs{66}\nwdocspar

\section{Licensing information for this software}

This software is available under the \gnu\ \textsc{gpl} 3.  To
minimize confusion, the following information is inserted into each
generated file.  The licensing and copyright conditions, obviously,
apply to this document as well.
\nwenddocs{}\nwbegincode{67}\sublabel{NW29Cep0-4UiXc2-1}\nwmargintag{{\nwtagstyle{}\subpageref{NW29Cep0-4UiXc2-1}}}\moddef{Copyright and licensing~{\nwtagstyle{}\subpageref{NW29Cep0-4UiXc2-1}}}\endmoddef\nwstartdeflinemarkup\nwusesondefline{\\{NW29Cep0-3NQVMk-1}\\{NW29Cep0-2ARW6h-1}\\{NW29Cep0-3NQVMk-4}\\{NW29Cep0-B4OnS-1}\\{NW29Cep0-HV7Dg-1}\\{NW29Cep0-15rBrk-1}\\{NW29Cep0-F2wuL-1}\\{NW29Cep0-4VkTYt-1}\\{NW29Cep0-15U98S-1}}\nwenddeflinemarkup
 # This file was automatically generated by noweb from implementation.rnw.
 # So you should probably not edit it.  Instead, please edit the source
 # file (implementation.rnw) and see implementation.pdf for documentation.
  
 # Copyright (c) 2011 by Gray Calhoun
 
 # This program is free software: you can redistribute it and/or modify
 # it under the terms of the GNU General Public License as published by
 # the Free Software Foundation, either version 3 of the License, or (at
 # your option) any later version.

 # This program is distributed in the hope that it will be useful, but
 # WITHOUT ANY WARRANTY; without even the implied warranty of
 # MERCHANTABILITY or FITNESS FOR A PARTICULAR PURPOSE. See the GNU
 # General Public License for more details.

 # For a copy of the GNU General Public License, please see
 # <http://www.r-project.org/Licenses/>.
\nwused{\\{NW29Cep0-3NQVMk-1}\\{NW29Cep0-2ARW6h-1}\\{NW29Cep0-3NQVMk-4}\\{NW29Cep0-B4OnS-1}\\{NW29Cep0-HV7Dg-1}\\{NW29Cep0-15rBrk-1}\\{NW29Cep0-F2wuL-1}\\{NW29Cep0-4VkTYt-1}\\{NW29Cep0-15U98S-1}}\nwendcode{}

\nwixlogsorted{c}{{bootindex.R}{NW29Cep0-B4OnS-1}{\nwixd{NW29Cep0-B4OnS-1}}}%
\nwixlogsorted{c}{{Calculate individual bootstrap means}{nw@notdef}{\nwixu{NW29Cep0-3NQVMk-6}}}%
\nwixlogsorted{c}{{Calculate individual test statistics}{NW29Cep0-2j4ZdE-1}{\nwixu{NW29Cep0-3NQVMk-6}\nwixd{NW29Cep0-2j4ZdE-1}}}%
\nwixlogsorted{c}{{Check and format \code{}rvar\edoc{} arguments}{NW29Cep0-KDAQP-1}{\nwixu{NW29Cep0-2Jbam7-1}\nwixd{NW29Cep0-KDAQP-1}}}%
\nwixlogsorted{c}{{Check inputs for \code{}CT\edoc{}}{NW29Cep0-12DbnH-1}{\nwixu{NW29Cep0-4VkTYt-1}\nwixd{NW29Cep0-12DbnH-1}}}%
\nwixlogsorted{c}{{Check inputs to \code{}Aggregate\edoc{}}{NW29Cep0-4U3W6i-1}{\nwixu{NW29Cep0-4VkTYt-3}\nwixd{NW29Cep0-4U3W6i-1}}}%
\nwixlogsorted{c}{{clarkwest.R}{NW29Cep0-2ARW6h-1}{\nwixd{NW29Cep0-2ARW6h-1}\nwixd{NW29Cep0-2ARW6h-2}}}%
\nwixlogsorted{c}{{Classes.R}{NW29Cep0-4VkTYt-1}{\nwixd{NW29Cep0-4VkTYt-1}\nwixd{NW29Cep0-4VkTYt-2}\nwixd{NW29Cep0-4VkTYt-3}\nwixd{NW29Cep0-4VkTYt-4}\nwixd{NW29Cep0-4VkTYt-5}}}%
\nwixlogsorted{c}{{Copyright and licensing}{NW29Cep0-4UiXc2-1}{\nwixu{NW29Cep0-3NQVMk-1}\nwixu{NW29Cep0-2ARW6h-1}\nwixu{NW29Cep0-3NQVMk-4}\nwixu{NW29Cep0-B4OnS-1}\nwixu{NW29Cep0-HV7Dg-1}\nwixu{NW29Cep0-15rBrk-1}\nwixu{NW29Cep0-F2wuL-1}\nwixu{NW29Cep0-4VkTYt-1}\nwixu{NW29Cep0-15U98S-1}\nwixd{NW29Cep0-4UiXc2-1}}}%
\nwixlogsorted{c}{{data frame of simulated series}{NW29Cep0-3E2M0d-1}{\nwixu{NW29Cep0-2Jbam7-1}\nwixd{NW29Cep0-3E2M0d-1}}}%
\nwixlogsorted{c}{{Define \code{}mixedbootstrap\edoc{} local variables}{NW29Cep0-3CIgww-1}{\nwixu{NW29Cep0-3NQVMk-6}\nwixd{NW29Cep0-3CIgww-1}}}%
\nwixlogsorted{c}{{Define block bootstrap function \code{}rbootindex\edoc{}}{NW29Cep0-3SDs62-1}{\nwixu{NW29Cep0-3NQVMk-6}\nwixd{NW29Cep0-3SDs62-1}}}%
\nwixlogsorted{c}{{Define faster \code{}findInterval\edoc{} function}{NW29Cep0-2cq75d-1}{\nwixu{NW29Cep0-HXXXl-1}\nwixd{NW29Cep0-2cq75d-1}}}%
\nwixlogsorted{c}{{dmw.R}{NW29Cep0-3NQVMk-1}{\nwixd{NW29Cep0-3NQVMk-1}\nwixd{NW29Cep0-3NQVMk-2}\nwixd{NW29Cep0-3NQVMk-3}\nwixd{NW29Cep0-3NQVMk-4}\nwixd{NW29Cep0-3NQVMk-5}\nwixd{NW29Cep0-3NQVMk-6}}}%
\nwixlogsorted{c}{{extract.R}{NW29Cep0-F2wuL-1}{\nwixd{NW29Cep0-F2wuL-1}}}%
\nwixlogsorted{c}{{Find critical values and determine which hypotheses are rejected}{NW29Cep0-4KyWj2-1}{\nwixu{NW29Cep0-HV7Dg-1}\nwixd{NW29Cep0-4KyWj2-1}}}%
\nwixlogsorted{c}{{Forecast with each individual model}{NW29Cep0-1jm0YB-1}{\nwixu{NW29Cep0-4VkTYt-4}\nwixd{NW29Cep0-1jm0YB-1}}}%
\nwixlogsorted{c}{{mccracken.R}{NW29Cep0-HXXXl-1}{\nwixd{NW29Cep0-HXXXl-1}}}%
\nwixlogsorted{c}{{mccrackendata.R}{NW29Cep0-18ASnA-1}{\nwixd{NW29Cep0-18ASnA-1}}}%
\nwixlogsorted{c}{{NAMESPACE}{NW29Cep0-15U98S-1}{\nwixd{NW29Cep0-15U98S-1}}}%
\nwixlogsorted{c}{{recursive-forecasts.R}{NW29Cep0-15rBrk-1}{\nwixd{NW29Cep0-15rBrk-1}}}%
\nwixlogsorted{c}{{Return aggregate single period forecast}{NW29Cep0-zSOm4-1}{\nwixu{NW29Cep0-4VkTYt-4}\nwixd{NW29Cep0-zSOm4-1}}}%
\nwixlogsorted{c}{{Return multiple-horizon aggregate forecasts}{NW29Cep0-2Zg6lR-1}{\nwixu{NW29Cep0-4VkTYt-4}\nwixd{NW29Cep0-2Zg6lR-1}}}%
\nwixlogsorted{c}{{rvar.R}{NW29Cep0-2Jbam7-1}{\nwixd{NW29Cep0-2Jbam7-1}}}%
\nwixlogsorted{c}{{stepm.R}{NW29Cep0-HV7Dg-1}{\nwixd{NW29Cep0-HV7Dg-1}}}%
\nwixlogsorted{i}{{\nwixident{Aggregate}}{Aggregate}}%
\nwixlogsorted{i}{{\nwixident{bootindex{\_}circularblock}}{bootindex:uncircularblock}}%
\nwixlogsorted{i}{{\nwixident{bootindex{\_}movingblock}}{bootindex:unmovingblock}}%
\nwixlogsorted{i}{{\nwixident{bootindex{\_}stationary}}{bootindex:unstationary}}%
\nwixlogsorted{i}{{\nwixident{clarkwest}}{clarkwest}}%
\nwixlogsorted{i}{{\nwixident{clarkwest{\_}calculation}}{clarkwest:uncalculation}}%
\nwixlogsorted{i}{{\nwixident{CT}}{CT}}%
\nwixlogsorted{i}{{\nwixident{dmw{\_}calculation}}{dmw:uncalculation}}%
\nwixlogsorted{i}{{\nwixident{dmw{\_}lambda}}{dmw:unlambda}}%
\nwixlogsorted{i}{{\nwixident{dmw{\_}mse}}{dmw:unmse}}%
\nwixlogsorted{i}{{\nwixident{extract{\_}predictors}}{extract:unpredictors}}%
\nwixlogsorted{i}{{\nwixident{extract{\_}target}}{extract:untarget}}%
\nwixlogsorted{i}{{\nwixident{findInterval{\_}presorted}}{findInterval:unpresorted}}%
\nwixlogsorted{i}{{\nwixident{HasMethod}}{HasMethod}}%
\nwixlogsorted{i}{{\nwixident{mccracken{\_}criticalvalue}}{mccracken:uncriticalvalue}}%
\nwixlogsorted{i}{{\nwixident{McCrackenData}}{McCrackenData}}%
\nwixlogsorted{i}{{\nwixident{mixedbootstrap}}{mixedbootstrap}}%
\nwixlogsorted{i}{{\nwixident{mixedwindow}}{mixedwindow}}%
\nwixlogsorted{i}{{\nwixident{mixedwindow{\_}calculation}}{mixedwindow:uncalculation}}%
\nwixlogsorted{i}{{\nwixident{PiBounds}}{PiBounds}}%
\nwixlogsorted{i}{{\nwixident{PiIntervals}}{PiIntervals}}%
\nwixlogsorted{i}{{\nwixident{predict.Aggregate}}{predict.Aggregate}}%
\nwixlogsorted{i}{{\nwixident{predict.CT}}{predict.CT}}%
\nwixlogsorted{i}{{\nwixident{rbootindex}}{rbootindex}}%
\nwixlogsorted{i}{{\nwixident{recursive{\_}forecasts}}{recursive:unforecasts}}%
\nwixlogsorted{i}{{\nwixident{rvar}}{rvar}}%
\nwixlogsorted{i}{{\nwixident{stepm}}{stepm}}%
\nwbegindocs{68}\nwdocspar
\bibliography{AllRefs}
\addcontentsline{toc}{section}{References}

\section*{Index}
\addcontentsline{toc}{section}{Index}

\subsection*{Functions and variables}
\addcontentsline{toc}{subsection}{Functions and variables}
\nowebindex

\subsection*{Code chunks}
\addcontentsline{toc}{subsection}{Code chunks}
\nowebchunks

\end{document}

% LocalWords:  mixedwindow dmw mixedbootstrap rbootindex blocklength stepm
% LocalWords:  extendedindex NAMESPACE clarkwest bootindex
\nwenddocs{}
